\documentclass[dvisvgm,hypertex,aspectratio=169]{beamer}
\usetheme{tmg}

\title{Lógica Proposicional}
\subtitle{Teoría y práctica desde cero}
\author{MathLike}
\date{\today}

\newenvironment{changemargin}{%
  \vspace*{-0.5cm}
  \begin{list}{}{%
    \setlength{\leftmargin}{-1cm}%
    \setlength{\rightmargin}{0pt}%
  }%
  \item[]}{\end{list}}
% Los roles son los siguientes:
% - Administrador (Solo MathLike puede tener este rol)
% - Miembro
% - Moderador (Solo moderadores del server pueden tener este rol)
% - Ayudante de [asignatura] (Solo ayudantes de esa asignatura pueden tener este rol)
% - Creador de contenido (Solo creadores de contenido pueden tener este rol)
\institute{Administrador de The Math Guys}
\titlegraphic{\includegraphics[height=0.25\paperheight]{TheMathGuysLogo.png}}
\discord{The math guys}
% Si no tiene YouTube o GitHub, simplemente borre el comando correspondiente
\youtube{mathlike}
\github{MathItYT}

\begin{document}
\begin{frame}
  \titlepage
\end{frame}
\begin{frame}{Motivación}
  \begin{itemize}[<+->]
    \item Tenemos la siguiente afirmación: \textit{Si MathLike roba, entonces MathLike se va a la cárcel}.
    \item Esto es cierto, pero, ¿lo será su recíproco?
    \item ¿Qué pasa si MathLike se va a la cárcel? ¿Es cierto que MathLike robó?
  \end{itemize}
\end{frame}
\begin{frame}{Motivación}
  \begin{itemize}[<+->]
    \item La verdad no necesariamente, pude haber sido acusado injustamente, haber matado a alguien, vendido cosas ilegales, etc.
    \item Es un ejemplo sencillo de lógica proposicional, ¡y que fácilmente hemos usado en la vida real!
    \item Cuando discutes con alguien, quizás no te des cuenta, pero te esfuerzas para que tus argumentos sean válidos.
    \item La lógica proposicional es la base de la lógica matemática, y es la que nos permite hacer razonamientos válidos.
    \item Podemos usarla también para modelar situaciones de la vida real, como estructurar las reglas de mi \textit{server} de Discord. \scalerel*{\twemoji{1f5e3}}{X}
    \item ¡Vamos a aprenderla! \scalerel*{\twemoji{nerd face}}{X}\scalerel*{\twemoji{261d}}{X}
  \end{itemize}
\end{frame}
\begin{frame}{Conceptos básicos}
  \begin{itemize}[<+->]
    \item \textbf{Proposición:} Una oración que puede ser verdadera o falsa, pero no ambas. Ejemplos: $2+2=4$, \textit{La Tierra es plana}.
    \item \textbf{Conectivos lógicos:} Son operaciones que se aplican a proposiciones para formar proposiciones más complejas. Por ejemplo, \textit{y}, \textit{o}, \textit{si... entonces...}, \textit{no es cierto que...}
    \item \textbf{Tabla de verdad:} Es una tabla que muestra todas las posibles combinaciones de valores de verdad de las proposiciones que la componen. $1$ es verdadero, $0$ es falso.

    \begin{table}[h]
      \centering
      \begin{tabular}{|c|c|c|}
        \hline
        $p$ & $q$ & $p \land q$ \\ \hline
        1 & 1 & 1 \\ \hline
        1 & 0 & 0 \\ \hline
        0 & 1 & 0 \\ \hline
        0 & 0 & 0 \\ \hline
      \end{tabular}
    \end{table}
  \end{itemize}
\end{frame}
\begin{frame}{Conceptos básicos}
  \begin{itemize}[<+->]
    \item \textbf{Proposición simple:} Una proposición que no contiene conectivos lógicos.
    \item \textbf{Proposición compuesta:} Una proposición que contiene al menos un conectivo lógico.
    \item \textbf{Conjunto proposicional:} Un conjunto de proposiciones simples. Por ejemplo, $\{p, q, r\}$, donde $p$ puede ser \textit{MathLike es admin}, $q$ puede ser \textit{MathLike es miembro}, y $r$ puede ser \textit{MathLike no es YouTuber} (¡falso!) \scalerel*{\twemoji{1f609}}{X}
    \item \textbf{Operador lógico:} Un símbolo que representa un conectivo lógico. Por ejemplo, $\land$ representa \textit{y} (binario), $\lor$ representa \textit{o} (binario), $\neg$ representa \textit{no} (unario).
    \item \textbf{Lenguaje formal:} Un lenguaje que se usa para expresar proposiciones y razonamientos de manera precisa y sin ambigüedades. Se usa un conjunto proposicional y otro de operadores lógicos.
  \end{itemize}
\end{frame}
\begin{frame}{Sintaxis}
  Es el estudio de la escritura de las proposiciones. En un lenguaje formal, se usan los siguientes símbolos:
  \begin{itemize}[<+->]
    \item \textbf{Variables proposicionales:} $p, q, r$. Representan proposiciones simples.
    \item \textbf{Operadores lógicos:} $\land, \lor, \neg, \implies, \iff$
    \item \textbf{Paréntesis:} $(, )$. Se usan para agrupar proposiciones compuestas.
  \end{itemize}
  \pause
  \begin{block}{Definición}
    Una fórmula bien formada (FWF) es una secuencia finita de símbolos de un lenguaje formal que cumple con las reglas de formación de dicho lenguaje.
  \end{block}
\end{frame}
\begin{frame}{Sintaxis -- Reglas de formación}
  \begin{itemize}[<+->]
    \begin{enumerate}
      \item Toda variable proposicional es una FWF.
      \item Si $\alpha$ es una FWF, entonces $\neg \alpha$ también lo es.
      \item Si $\alpha$ y $\beta$ son FWF, entonces $(\alpha \land \beta)$, $(\alpha \lor \beta)$, $(\alpha \implies \beta)$ y $(\alpha \iff \beta)$ también lo son.
      \item Es el conjunto más pequeño que cumple con las reglas anteriores.
    \end{enumerate}
    \item \textbf{Notación:} Se usan paréntesis para evitar ambigüedades. Por ejemplo, $p \land q \lor r$ puede ser $(p \land q) \lor r$ o $p \land (q \lor r)$.
    \item \textbf{Precedencia:} $\neg$ tiene mayor precedencia que $\land$, $\land$ tiene mayor precedencia que $\lor$, y $\lor$ tiene mayor precedencia que $\implies$ y $\iff$.
  \end{itemize}
\end{frame}
\begin{frame}{Sintaxis -- Ejemplos}
  \begin{itemize}[<+->]
    \item $(p \land q) \lor r$ es una FWF.
    \item $\neg p \land q$ es una FWF.
    \item $p \land \neg q$ es una FWF.
    \item $p \land q \implies r$ es una FWF.
    \item $p \land (q \lor r)$ es una FWF.
    \item $p \land q \lor r \land s$ no es una FWF.
    \item $p \land \neg q \land r$ no es una FWF.
  \end{itemize}
\end{frame}
\begin{frame}{Semántica}
  Es el estudio del significado de las proposiciones. En la lógica proposicional, se usan las tablas de verdad para determinar el valor de verdad de una proposición compuesta.
  \begin{itemize}[<+->]
    \item \textbf{Interpretación:} Asignación de valores de verdad a las variables proposicionales.
    \item \textbf{Valor de verdad:} Valor que toma una proposición compuesta dada una interpretación.
    \item \textbf{Modelo:} Interpretación que hace que una proposición compuesta sea verdadera.
    \item \textbf{Satisfacibilidad:} Propiedad de una proposición compuesta que puede ser verdadera dada una interpretación.
    \item \textbf{Tautología:} Proposición compuesta que es verdadera para todas las interpretaciones.
    \item \textbf{Contradicción:} Proposición compuesta que es falsa para todas las interpretaciones.
  \end{itemize}
\end{frame}
\begin{frame}{Semántica -- Tablas de verdad}
  \begin{itemize}[<+->]
    \item \textbf{Ejemplo:} Dada la proposición $p \land q$, su tabla de verdad es:
    \begin{table}[h]
      \centering
      \begin{tabular}{|c|c|c|}
        \hline
        $p$ & $q$ & $p \land q$ \\ \hline
        1 & 1 & 1 \\ \hline
        1 & 0 & 0 \\ \hline
        0 & 1 & 0 \\ \hline
        0 & 0 & 0 \\ \hline
      \end{tabular}
    \end{table}
    \item \textbf{Ejemplo:} Dada la proposición $p \lor q$, su tabla de verdad es:
    \begin{table}[h]
      \centering
      \begin{tabular}{|c|c|c|}
        \hline
        $p$ & $q$ & $p \lor q$ \\ \hline
        1 & 1 & 1 \\ \hline
        1 & 0 & 1 \\ \hline
        0 & 1 & 1 \\ \hline
        0 & 0 & 0 \\ \hline
      \end{tabular}
    \end{table}
  \end{itemize}
\end{frame}
\begin{frame}{Semántica -- Ejemplos de tablas de verdad}
  \begin{itemize}[<+->]
    \item Dada la proposición $p \land q \lor r$, su tabla de verdad es:
    \begin{table}[h]
      \centering
      \begin{tabular}{|c|c|c|c|c|}
        \hline
        $p$ & $q$ & $r$ & $q \lor r$ & $p \land (q \lor r)$ \\ \hline
        1 & 1 & 1 & 1 & 1 \\ \hline
        1 & 1 & 0 & 1 & 1 \\ \hline
        1 & 0 & 1 & 1 & 1 \\ \hline
        1 & 0 & 0 & 0 & 0 \\ \hline
        0 & 1 & 1 & 1 & 0 \\ \hline
        0 & 1 & 0 & 1 & 0 \\ \hline
        0 & 0 & 1 & 1 & 0 \\ \hline
        0 & 0 & 0 & 0 & 0 \\ \hline
      \end{tabular}
    \end{table}
    \item Dada la proposición $p \land \neg q$, su tabla de verdad es:
    \begin{table}[h]
      \centering
      \begin{tabular}{|c|c|c|}
        \hline
        $p$ & $q$ & $p \land \neg q$ \\ \hline
        1 & 1 & 0 \\ \hline
        1 & 0 & 1 \\ \hline
        0 & 1 & 0 \\ \hline
        0 & 0 & 0 \\ \hline
      \end{tabular}
    \end{table}
  \end{itemize}
\end{frame}
\begin{frame}{Semántica -- Tablas de verdad notables}
  \begin{itemize}[<+->]
    \item \textbf{Negación:} Dada la proposición $\neg p$, su tabla de verdad es:
    \begin{table}[h]
      \centering
      \begin{tabular}{|c|c|}
        \hline
        $p$ & $\neg p$ \\ \hline
        1 & 0 \\ \hline
        0 & 1 \\ \hline
      \end{tabular}
    \end{table}
    \item \textbf{Conjunción:} Dada la proposición $p \land q$, su tabla de verdad es:
    \begin{table}[h]
      \centering
      \begin{tabular}{|c|c|c|}
        \hline
        $p$ & $q$ & $p \land q$ \\ \hline
        1 & 1 & 1 \\ \hline
        1 & 0 & 0 \\ \hline
        0 & 1 & 0 \\ \hline
        0 & 0 & 0 \\ \hline
      \end{tabular}
    \end{table}
  \end{itemize}
\end{frame}
\begin{frame}{Semántica -- Tablas de verdad notables}
  \begin{itemize}[<+->]
    \item \textbf{Disyunción:} Dada la proposición $p \lor q$, su tabla de verdad es:
    \begin{table}[h]
      \centering
      \begin{tabular}{|c|c|c|}
        \hline
        $p$ & $q$ & $p \lor q$ \\ \hline
        1 & 1 & 1 \\ \hline
        1 & 0 & 1 \\ \hline
        0 & 1 & 1 \\ \hline
        0 & 0 & 0 \\ \hline
      \end{tabular}
    \end{table}
    \item \textbf{Implicación:} Dada la proposición $p \implies q$, su tabla de verdad es:
    \begin{table}[h]
      \centering
      \begin{tabular}{|c|c|c|}
        \hline
        $p$ & $q$ & $p \implies q$ \\ \hline
        1 & 1 & 1 \\ \hline
        1 & 0 & 0 \\ \hline
        0 & 1 & 1 \\ \hline
        0 & 0 & 1 \\ \hline
      \end{tabular}
    \end{table}
  \end{itemize}
\end{frame}
\begin{frame}{Semántica -- Tablas de verdad notables}
  \begin{itemize}[<+->]
    \item \textbf{Doble implicación:} Dada la proposición $p \iff q$, su tabla de verdad es:
    \begin{table}[h]
      \centering
      \begin{tabular}{|c|c|c|}
        \hline
        $p$ & $q$ & $p \iff q$ \\ \hline
        1 & 1 & 1 \\ \hline
        1 & 0 & 0 \\ \hline
        0 & 1 & 0 \\ \hline
        0 & 0 & 1 \\ \hline
      \end{tabular}
    \end{table}
  \end{itemize}
\end{frame}
\begin{frame}{Semántica -- Combinatoria}
  \begin{itemize}[<+->]
    \item ¿Cuántas proposiciones compuestas se pueden formar con $n$ variables proposicionales?
    \item Sabemos que hay $2^n$ interpretaciones para $n$ variables proposicionales.
    \item Cada interpretación puede ser verdadera o falsa, por lo que hay $2^{2^n}$ proposiciones compuestas posibles.
    \item Por ejemplo, con $n = 2$, hay $2^{2^2} = 2^4 = 16$ proposiciones compuestas posibles.
  \end{itemize}
\end{frame}
\begin{frame}{Equivalencia lógica}
  \begin{itemize}[<+->]
    \item \textbf{Definición:} Dadas dos proposiciones $p$ y $q$, se dice que son lógicamente equivalentes si tienen la misma tabla de verdad.
    \item \textbf{Notación:} $p \equiv q$.
    \item \textbf{Propiedades:}
    \begin{enumerate}
      \item \textbf{Reflexiva:} $p \equiv p$.
      \item \textbf{Simétrica:} Si $p \equiv q$, entonces $q \equiv p$.
      \item \textbf{Transitiva:} Si $p \equiv q$ y $q \equiv r$, entonces $p \equiv r$.
      \item \textbf{Teorema de la tautología:} $p \equiv q$ si y solo si $(p \iff q)$ es una tautología.
    \end{enumerate}
  \end{itemize}
\end{frame}
\begin{frame}{Equivalencia lógica -- Ejemplos}
  \begin{itemize}[<+->]
    \item Dadas las proposiciones $p \land q$ y $q \land p$, son lógicamente equivalentes.
    \item Dadas las proposiciones $p \lor q$ y $q \lor p$, son lógicamente equivalentes.
    \item Dadas las proposiciones $p \land (q \lor r)$ y $(p \land q) \lor (p \land r)$, son lógicamente equivalentes.
    \item Dadas las proposiciones $p \lor (q \land r)$ y $(p \lor q) \land (p \lor r)$, no son lógicamente equivalentes.
  \end{itemize}
\end{frame}
\begin{frame}{Equivalencia lógica -- Leyes de Morgan}
  \begin{itemize}[<+->]
    \item \textbf{Ley de Morgan para la conjunción:} $\neg (p \land q) \equiv \neg p \lor \neg q$.
    \item \textbf{Ley de Morgan para la disyunción:} $\neg (p \lor q) \equiv \neg p \land \neg q$.
  \end{itemize}
\end{frame}
\begin{frame}{Equivalencia lógica -- Distributividad}
  \begin{itemize}[<+->]
    \item \textbf{Propiedad distributiva de la conjunción sobre la disyunción:} $p \land (q \lor r) \equiv (p \land q) \lor (p \land r)$.
    \item \textbf{Propiedad distributiva de la disyunción sobre la conjunción:} $p \lor (q \land r) \equiv (p \lor q) \land (p \lor r)$.
  \end{itemize}
\end{frame}
\begin{frame}{Equivalencia lógica -- Transitividad}
  \begin{itemize}[<+->]
    \item Si $p \iff q$ y $q \iff r$, entonces $p \iff r$.
    \item Si $p \implies q$ y $q \implies r$, entonces $p \implies r$.
    \item Si $p \land q$ y $q \land r$, entonces $p \land r$.
    \item Si $p \lor q$ y $q \lor r$, entonces $p \lor r$.
  \end{itemize}
\end{frame}
\begin{frame}{Equivalencia lógica -- Simplificación}
  \begin{itemize}[<+->]
    \item \textbf{Propiedad de la identidad:} $p \land \top \equiv p$ y $p \lor \bot \equiv p$.
    \item \textbf{Propiedad de la dominación:} $p \land \bot \equiv \bot$ y $p \lor \top \equiv \top$.
    \item \textbf{Propiedad de la doble negación:} $\neg \neg p \equiv p$.
  \end{itemize}
\end{frame}
\begin{frame}{Equivalencia lógica -- Simplificación}
  \begin{itemize}[<+->]
    \item \textbf{Propiedad de la idempotencia:} $p \land p \equiv p$ y $p \lor p \equiv p$.
    \item \textbf{Propiedad de la complementación:} $p \land \neg p \equiv \bot$ y $p \lor \neg p \equiv \top$.
    \item \textbf{Propiedad de la absorción:} $p \land (p \lor q) \equiv p$ y $p \lor (p \land q) \equiv p$.
  \end{itemize}
\end{frame}
\begin{frame}{Problema real}
  MathLike quiere un código de conducta para su \textit{server} de Discord. Quiere que se cumplan las siguientes reglas:

  \begin{enumerate}[<+->]
    \item Si un usuario hace \textit{spam} o \textit{trolea}, entonces será \textit{baneado}.
    \item Si un usuario hace \textit{flood}, entonces será \textit{warned} y \textit{muteado}.
    \item Si un usuario hace \textit{flood} y \textit{trolea}, entonces será \textit{muteado}.
    \item Si un usuario hace \textit{spam} y \textit{flood}, entonces será \textit{baneado}.
    \item Si un usuario envía \textit{NSFW}, entonces será \textit{baneado}.
  \end{enumerate}

  ¿Cómo podemos expresar estas reglas en lógica proposicional? ¿Se puede simplificar?
\end{frame}
\end{document}