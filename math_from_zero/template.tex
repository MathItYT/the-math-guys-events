\documentclass[dvisvgm,hypertex,aspectratio=169]{beamer}
\usetheme{tmg}

\title{Matemáticas desde cero -- Clase 1}
\subtitle{Desde otra perspectiva: Lógica}
\author{MathLike}
\date{Abril de 2024}

\newenvironment{changemargin}{%
  \vspace*{-0.5cm}
  \begin{list}{}{%
    \setlength{\leftmargin}{-1cm}%
    \setlength{\rightmargin}{0pt}%
  }%
  \item[]}{\end{list}}
% Los roles son los siguientes:
% - Administrador (Solo MathLike puede tener este rol)
% - Miembro
% - Moderador (Solo moderadores del server pueden tener este rol)
% - Ayudante de [asignatura] (Solo ayudantes de esa asignatura pueden tener este rol)
% - Creador de contenido (Solo creadores de contenido pueden tener este rol)
\institute{Administrador de The Math Guys}
\titlegraphic{\includegraphics[height=0.25\paperheight]{TheMathGuysLogo.png}}
\discord{The math guys}
% Si no tiene YouTube o GitHub, simplemente borre el comando correspondiente
\youtube{mathlike}
\github{MathItYT}

\begin{document}
\begin{frame}
  \titlepage
\end{frame}
\begin{frame}{Motivación}
  Desde la escuela siempre nos han enseñado que contar y sumar es lo más básico
  de las matemáticas, pero, ¿hay algo más básico que decir $\VAR{a}+\VAR{b}=\VAR{result}$
  o que en \VAR{apples} hay \VAR{apple_count} manzanas?
\end{frame}
\begin{frame}{¿Qué necesitamos antes?}
  Las matemáticas son exactas y consistentes, es decir, no hay lugar para la
  ambigüedad o la contradicción. \pause{}Para evitar las dos cosas, necesitamos darle
  estructura a nuestros razonamientos. De esto se encarga la \alert{lógica}.
\end{frame}
\begin{frame}{¿Es válido este argumento?}
  \begin{block}{Argumento}
    \pause
    \[
    \begin{array}{ c c }
      &\text{Todas las personas son mortales} \pause\\
      &\text{Sócrates es una persona} \pause\\
      \cline{2-2}
      \therefore &\text{Sócrates es mortal}
    \end{array}
    \]
  \end{block}
\end{frame}
\begin{frame}{¿Qué es la lógica?}
  Intuitivamente, para nosotros sí es válido el argumento anterior. Sócrates es una persona, y toda persona es mortal, o sea que Sócrates debe ser mortal.

  \pause Pero, ¿cómo podemos formalizar esto? De esto se encarga la lógica.
\end{frame}
\begin{frame}{Proposición}
  Una proposición es una afirmación que puede ser verdadera o falsa, pero no ambas a la vez ni ninguna.

  \begin{block}{Ejemplos}
    \begin{itemize}
      \item $2+2=4$ es proposición, específicamente verdadera.
      \item $1+5=98$ es proposición, específicamente falsa.
      \item $x+3=5$ no es proposición, ya que depende del valor de $x$.
      \item "¿Qué hora es?" no es proposición, ya que no se puede determinar el valor de verdad de una pregunta.
      \item "¡Hola!" no es proposición, ya que no es ni verdadera ni falsa.
      \item "Haz la tarea" no es proposición, ya que no es ni verdadera ni falsa.
    \end{itemize}
  \end{block}
\end{frame}
\begin{frame}{Formalizando proposiciones}
  \begin{block}{Definición}
    Un lenguaje $\mathcal{L}(P, S_1, S_2)$ sobre un conjunto de proposiciones simples $P$ es un conjunto de expresiones que nos permiten construir proposiciones a partir de las proposiciones en $P$
    usando los conectivos lógicos unarios en $S_1$ y binarios en $S_2$. Es decir, $\mathcal{L}(P, S_1, S_2)$ es la clase más pequeña que cumple:

    \pause
    $$p\in P\implies p\in\mathcal{L}(P, S_1, S_2)$$
    \begin{align*}
      &(p,q\in\mathcal{L}(P, S_1, S_2))\land(*_1\in S_1)\land(*_2\in S_2) \\
      \implies&(*_1(p)\in\mathcal{L}(P, S_1, S_2))\land((p*_2q)\in\mathcal{L}(P, S_1, S_2))
    \end{align*}
  \end{block}
  \pause
  En el primer caso, $p$ es una proposición simple como $2+2=5$ (falsa) o Santiago es la capital de Chile (verdadera).
\end{frame}
\begin{frame}{Formalizando proposiciones}
  En el segundo caso, $p$ es una proposición arbitraria en el lenguaje como $2+2=4$ (verdadera) o Santiago es la capital de Chile y $1+1=3$ (falsa).

  \pause
  Los paréntesis sirven para agrupar y evitar ambigüedades.
\end{frame}
\begin{frame}{Ejercicio}
  Sean $p, q, r\in P$, $\neg\in S_1$ y $\land,\lor,\iff\in S_2$. Demuestre que $(\neg p\land(q\lor r))\iff((p\land q)\lor(p\land r))\in\mathcal{L}(P, S_1, S_2)$. (Nótese que la expresión no está rodeada en paréntesis, porque no es ambiguo, pero sigue siendo válido).

  \pause
  \textbf{Demostración:} Por definición, $p, q, r\in\mathcal{L}(P, S_1, S_2)$. Por lo tanto, $\neg p\in\mathcal{L}(P, S_1, S_2)$ y $q\lor r\in\mathcal{L}(P, S_1, S_2)$. Luego, $\neg p\in\mathcal{L}(P, S_1, S_2)$ y $q\lor r\in\mathcal{L}(P, S_1, S_2)$, por lo que $(\neg p\land(q\lor r))\in\mathcal{L}(P, S_1, S_2)$.

  \pause De manera similar, $p\land q\in\mathcal{L}(P, S_1, S_2)$ y $p\land r\in\mathcal{L}(P, S_1, S_2)$, por lo que $(p\land q)\lor(p\land r)\in\mathcal{L}(P, S_1, S_2)$.
  
  \pause Finalmente, $(\neg p\land(q\lor r))\iff((p\land q)\lor(p\land r))\in\mathcal{L}(P, S_1, S_2)$. $\blacksquare$
\end{frame}
\begin{frame}{Valuación}
  \begin{block}{Definición}
    Una valuación $\sigma:\mathcal{L}(P, S_1, S_2)\to\{0,1\}$ es una función que asigna a cada proposición compuesta un valor de verdad, dependiendo de los valores de verdad de las proposiciones simples.
  \end{block}
\end{frame}
\begin{frame}{Definición de operadores}
  \pause
  \begin{block}{Definición}
    $\land$ es un operador que satisface $\sigma(p\land q)=1$ si y solo si $\sigma(p)=1$ y $\sigma(q)=1$.
  \end{block}
  \pause
  \begin{block}{Definición}
    $\lor$ es un operador que satisface $\sigma(p\lor q)=1$ si y solo si $\sigma(p)=1$ o $\sigma(q)=1$.
  \end{block}
  \pause
  \begin{block}{Definición}
    $\neg$ es un operador que satisface $\sigma(\neg p)=1$ si y solo si $\sigma(p)=0$.
  \end{block}
  \pause
  \begin{block}{Definición}
    $\iff$ es un operador que satisface $\sigma(p\iff q)=1$ si y solo si $\sigma(p)=\sigma(q)$.
  \end{block} 
  \pause
  \begin{block}{Definición}
    $\implies$ es un operador que satisface $\sigma(p\implies q)=\sigma(\neg p\lor q)$.
  \end{block}
\end{frame}
\begin{frame}{Tablas de verdad}
  \begin{block}{Definición}
    Una tabla de verdad es una tabla que muestra los valores de verdad de una proposición compuesta para todas las posibles combinaciones de valores de verdad de las proposiciones simples.

    \pause
    Por ejemplo, la tabla de verdad de $p\land q$ es:
    \begin{center}
      \begin{tabular}{ |c|c|c|c| }
        \hline
        $\sigma$ & $p$ & $q$ & $p\land q$ \\
        \hline
        $\sigma_1$ & 1 & 1 & 1 \\
        $\sigma_2$ & 1 & 0 & 0 \\
        $\sigma_3$ & 0 & 1 & 0 \\
        $\sigma_4$ & 0 & 0 & 0 \\
        \hline
      \end{tabular}
    \end{center}
  \end{block}
\end{frame}
\begin{frame}{Ejercicio}
  \begin{block}{Ejercicio}
    Encuentre la tabla de verdad de $p\lor q$.

    \pause
    \begin{center}
      \begin{tabular}{ |c|c|c|c| }
        \hline
        $\sigma$ & $p$ & $q$ & $p\lor q$ \\
        \hline
        $\sigma_1$ & 1 & 1 & 1 \\
        $\sigma_2$ & 1 & 0 & 1 \\
        $\sigma_3$ & 0 & 1 & 1 \\
        $\sigma_4$ & 0 & 0 & 0 \\
        \hline
      \end{tabular}
    \end{center}
  \end{block}
\end{frame}
\begin{frame}{Ejercicio}
  \begin{block}{Ejercicio}
    Encuentre la tabla de verdad de $\neg p$.

    \pause
    \begin{center}
      \begin{tabular}{ |c|c|c| }
        \hline
        $\sigma$ & $p$ & $\neg p$ \\
        \hline
        $\sigma_1$ & 1 & 0 \\
        $\sigma_2$ & 0 & 1 \\
        \hline
      \end{tabular}
    \end{center}
  \end{block}
\end{frame}
\begin{frame}{Ejercicio}
  \begin{block}{Ejercicio}
    Encuentre la tabla de verdad de $p\implies q$.

    \pause
    \begin{center}
      \begin{tabular}{ |c|c|c|c| }
        \hline
        $\sigma$ & $p$ & $q$ & $p\implies q$ \\
        \hline
        $\sigma_1$ & 1 & 1 & 1 \\
        $\sigma_2$ & 1 & 0 & 0 \\
        $\sigma_3$ & 0 & 1 & 1 \\
        $\sigma_4$ & 0 & 0 & 1 \\
        \hline
      \end{tabular}
    \end{center}
  \end{block}
\end{frame}
\begin{frame}{Ejercicio}
  \begin{block}{Ejercicio}
    Encuentre la tabla de verdad de $p\iff q$.

    \pause
    \begin{center}
      \begin{tabular}{ |c|c|c|c| }
        \hline
        $\sigma$ & $p$ & $q$ & $p\iff q$ \\
        \hline
        $\sigma_1$ & 1 & 1 & 1 \\
        $\sigma_2$ & 1 & 0 & 0 \\
        $\sigma_3$ & 0 & 1 & 0 \\
        $\sigma_4$ & 0 & 0 & 1 \\
        \hline
      \end{tabular}
    \end{center}
  \end{block}
\end{frame}
\begin{frame}{Ejercicio}
  \begin{block}{Ejercicio}
    Encuentre la tabla de verdad de $p\land(q\lor r)$.

    \pause
    \begin{center}
      \begin{tabular}{ |c|c|c|c|c|c| }
        \hline
        $\sigma$ & $p$ & $q$ & $r$ & $q\lor r$ & $p\land(q\lor r)$ \\
        \hline
        $\sigma_1$ & 1 & 1 & 1 & 1 & 1 \\
        $\sigma_2$ & 1 & 1 & 0 & 1 & 1 \\
        $\sigma_3$ & 1 & 0 & 1 & 1 & 1 \\
        $\sigma_4$ & 1 & 0 & 0 & 0 & 0 \\
        $\sigma_5$ & 0 & 1 & 1 & 1 & 0 \\
        $\sigma_6$ & 0 & 1 & 0 & 1 & 0 \\
        $\sigma_7$ & 0 & 0 & 1 & 1 & 0 \\
        $\sigma_8$ & 0 & 0 & 0 & 0 & 0 \\
        \hline
      \end{tabular}
    \end{center}
  \end{block}
\end{frame}
\begin{frame}{¿Qué más sigue?}
  \begin{block}{Próxima clase}
    \begin{itemize}
      \item Introducción a la teoría de conjuntos.
      \item Operaciones con conjuntos.
      \item Propiedades de las operaciones con conjuntos.
    \end{itemize}
  \end{block}
\end{frame}
\begin{frame}
  \begin{changemargin}
    \includemedia[
      width=1\paperwidth,
      height=1\paperheight,
      autoplay,
      muted,
      loop
    ]{}{ThanksForWatching.mp4}
  \end{changemargin}
\end{frame}
\end{document}